% !TEX root = frideswide.tex

\chapter{Sigla and Abbreviations}

\begin{raggedright}
\setkomafont{descriptionlabel}{}

\begin{description}[leftmargin=!,labelwidth=4em, itemsep=0pt, parsep=0pt]
\item[\emph{B}]
Oxford, Balliol College, \textsc{MS} 228
\item[\emph{C}]
Cambridge, Gonville and Caius College, \textsc{MS} 129/67
\item[\emph{D}]
Oxford, Bodleian Library, \textsc{MS} Digby 177
\item[\emph{G}]
Gotha, Forschungsbibliothek, \textsc{MS} Memb.~I~81
\item[\emph{L}]
London, British Library, Lansdowne \textsc{MS} 436
\item[\emph{M}]
Oxford, Bodleian Library, \textsc{MS} Laud misc. 114
\item[\emph{N}]
London, British Library, Cotton \textsc{MS} Nero E.~\textsc{i}/2
\item[\emph{P}]
Paris, Bibliothèque nationale de France, \textsc{MS} Latin 5320
\item[\emph{T}]
Cambridge, Trinity College, \textsc{MS} B.14.37
\item[\emph{W}]
Worcester, Cathedral Library, \textsc{MS} Q.86
\end{description}

\begin{description}[leftmargin=!,labelwidth=4em, itemsep=0pt, parsep=0pt]
\item[\emph{add.}]
\textlatin{\emph{addidit}}
\item[\emph{ante corr.}]
\textlatin{\emph{ante correctionem}}
\item[\emph{\textsc{BHL}}]
\textlatin{\emph{Bibliotheca hagiographica latina}}
\item[\emph{in marg.}]
\textlatin{\emph{in margine}}
\item[\emph{om.}]
\textlatin{\emph{omisit}}
\item[\emph{sup.~l.}]
\textlatin{\emph{supra lineam}}
\item[Vulg.]
Vulgate
\item[⟨\,\ldots{}\,⟩]
\textlatin{\emph{addendum}}
\end{description}
\end{raggedright}
