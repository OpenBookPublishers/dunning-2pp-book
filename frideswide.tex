% !TeX program = lualatex
\documentclass[
  paper=234mm:156mm, % Royal size
  DIV=classic, % margin grid calculation
  % BCOR=5mm, % binding correction
  fontsize=11pt, % base font size
  headings=normal, % base heading sizes/spacing
  parskip=never % no space between paragraphs
]{scrbook}

% Font settings
\usepackage{fontspec,microtype}
\defaultfontfeatures{Scale=MatchLowercase}
\defaultfontfeatures[\rmfamily]{Ligatures=TeX,Scale=1}

\setmainfont{ArnoPro}[
  % Path = {./fonts/},
  % Extension = {.otf},
  UprightFont = {*-Regular},
  ItalicFont = {*-Italic},
  BoldFont = {*-Smbd},
  BoldItalicFont = {*-SmbdItalic},
  Numbers={Lowercase,Proportional},
  UprightFeatures={
    SizeFeatures={
      {Size={-8.5},      Font=*-Caption},
      {Size={8.6-10.99}, Font=*-SmText},
      {Size={11-14},     Font=*-Regular},
      {Size={14.1-21.59},Font=*-Subhead},
      {Size={21.6-35.99},Font=*-Display},
      {Size={36-},       Font=*-LightDisplay}
    },
  },
  ItalicFeatures={
    SizeFeatures={
      {Size={-8.5},      Font=*-ItalicCaption},
      {Size={8.6-10.99}, Font=*-ItalicSmText},
      {Size={11-14},     Font=*-Italic},
      {Size={14.1-21.59},Font=*-ItalicSubhead},
      {Size={21.6-35.99},Font=*-ItalicDisplay},
      {Size={36-},       Font=*-LightItalicDisplay}
    },
  },
  BoldFeatures={
    SizeFeatures={
      {Size={-8.5},      Font=*-SmbdCaption},
      {Size={8.6-10.99}, Font=*-SmbdSmText},
      {Size={11-14},     Font=*-Smbd},
      {Size={14.1-21.59},Font=*-SmbdSubhead},
      {Size={21.6-},     Font=*-SmbdDisplay}
    },
  },
  BoldItalicFeatures={
    SizeFeatures={
      {Size={-8.5},      Font=*-SmbdItalicCaption},
      {Size={8.6-10.99}, Font=*-SmbdItalicSmText},
      {Size={11-14},     Font=*-SmbdItalic},
      {Size={14.1-21.59},Font=*-SmbdItalicSubhead},
      {Size={21.6-},     Font=*-SmbdItalicDisplay}
    },
  }
]

% use Junicode for other characters
\usepackage{newunicodechar}
\newfontfamily{\fallbackfont}{Junicode}
\DeclareTextFontCommand{\textfallback}{\fallbackfont}
\newunicodechar{⟨}{\textfallback{⟨}}
\newunicodechar{⟩}{\textfallback{⟩}}
\newunicodechar{⟦}{\textfallback{⟦}}
\newunicodechar{⟧}{\textfallback{⟧}}
\newunicodechar{✠}{\textfallback{✠}}
\newunicodechar{˷}{\textfallback{˷}}
\newunicodechar{⸝}{\textfallback{⸝}}
\newunicodechar{⸵}{\textfallback{}} % MUFI PUA

% use OpenType superscripts
\usepackage{realscripts}

% use all small caps for \textsc via OpenType
\let\mixtextsc\textsc
\renewcommand{\textsc}[1]{\mixtextsc{\addfontfeatures{Letters=UppercaseSmallCaps}{#1}}}

\setlength{\emergencystretch}{3em} % prevent overfull lines

% Adjust headings
\setkomafont{sectioning}{\normalfont}
\setkomafont{subsection}{\itshape}
\setkomafont{subsubsection}{\itshape}
\setkomafont{paragraph}{\scshape\addfontfeatures{Letters=UppercaseSmallCaps}}
\setkomafont{descriptionlabel}{\normalfont\scshape\addfontfeatures{Letters=UppercaseSmallCaps}}

\setcounter{secnumdepth}{-\maxdimen} % remove section numbering

% Move footnote marker into left margin
\deffootnote{0em}{1.6em}{\thefootnotemark\enskip}

\setfootnoterule{0pt} % Remove footnote rule

\usepackage{enumitem} % custom list formatting
\providecommand{\tightlist}{% lists without extra space
  \setlength{\itemsep}{0pt}\setlength{\parskip}{0pt}}

\usepackage{longtable,booktabs} % tables

% Manual bibliography format
\makeatletter
\renewenvironment{thebibliography}[1]
{\list{}%
  {\setlength{\labelwidth}{0pt}%
    \setlength{\labelsep}{0pt}%
    \setlength{\leftmargin}{\parindent}%
    \setlength{\itemindent}{-\parindent}%
    \setlength{\itemsep}{0pt}%
    \setlength{\parskip}{0pt}%
    \@openbib@code
    \usecounter{enumiv}}%
  \sloppy
  \sfcode`\.\@m}
{\def\@noitemerr
  {\@latex@warning{Empty `thebibliography' environment}}%
  \endlist}
\makeatother

% PDF links
\usepackage[open]{bookmark}
\hypersetup{
  unicode,
  hidelinks,
  pdftitle={Two Priors and a Princess: St Frideswide in Twelfth-Century Oxford},
  pdfauthor={Andrew Dunning with Benedicta Ward},
}
\usepackage{xurl}
\urlstyle{same}  % do not use monospace font for URLs

% Fix problems with old-style figures in URLs with LuaTeX: see <https://github.com/lualatex/luaotfload/issues/204>
\makeatletter
\g@addto@macro\UrlSpecials{%
  \do\0{\mbox{\UrlFont\char`\0}}%
  \do\1{\mbox{\UrlFont\char`\1}}%
  \do\2{\mbox{\UrlFont\char`\2}}%
  \do\3{\mbox{\UrlFont\char`\3}}%
  \do\4{\mbox{\UrlFont\char`\4}}%
  \do\5{\mbox{\UrlFont\char`\5}}%
  \do\6{\mbox{\UrlFont\char`\6}}%
  \do\7{\mbox{\UrlFont\char`\7}}%
  \do\8{\mbox{\UrlFont\char`\8}}%
  \do\9{\mbox{\UrlFont\char`\9}}%
}
\makeatother

\usepackage{graphicx} % enable image insertion


% Reledmac setup

% apparatus: apparatus criticus = A, commentary = B
\usepackage[series={A,B},noeledsec,noend,nofamiliar,noledgroup,nopbinverse]{reledmac}
\Xnotenumfont{\bfseries} % note line references in bold
\Xnumberonlyfirstinline % do not repeat line numbers
\Xnumberonlyfirstintwolines % take ranges into account for repetition
\AtBeginDocument{% allow notes to take up more of page
   \Xmaxhnotes{0.9\textheight}
}
\Xnonbreakableafternumber % disallow breaks after line references
% character space in place of lemma separator
\Xinplaceoflemmaseparator{\fontdimen2\font plus\fontdimen3\font minus\fontdimen4\font} 

% Change font size for marginal text
\renewcommand{\numlabfont}{\normalfont\footnotesize}
\renewcommand{\ledlsnotefontsetup}{\raggedleft\footnotesize}
\renewcommand{\ledrsnotefontsetup}{\raggedright\footnotesize}

% sidenotes and line numbering
\sidenotemargin{outer}
\leftnoteupfalse \rightnoteupfalse % Align sidenotes with first, not last line
\linenummargin{outer}

% horizontal distance of line numbers/sidenotes from text
\setlength{\linenumsep}{10pt}
\setlength{\ledlsnotesep}{10pt}
\setlength{\ledrsnotesep}{10pt}

% Arrange critical apparatus as one paragraph per page
\Xarrangement[A]{paragraph}
\Xsymlinenum[A]{\textbar{}}
\Xafternote[A]{1em plus.4em minus.4em}
\Xaftersymlinenum[A]{1em plus.4em minus.4em}

% Arrange commentary as paragraphs grouped by line
\Xlemmafont[B]{\itshape}
\Xlemmaseparator[B]{\emph{:}}
\Xwrapcontent[B]{\foreignlanguage{canadian}} % set commentary text to English
\Xparindent[B]
\Xinplaceofnumber[B]{0pt}
\Xgroupbyline[B]
\Xgroupbylineseparetwolines[B]
\Xsymlinenum[B]{◆}
\Xafternote[B]{0.6em plus.4em minus.4em}
\Xaftersymlinenum[B]{0.6em plus.4em minus.4em}

% Prefixes for cross references
\setapprefprefixsingle{line }
\setapprefprefixmore{lines }
\setSErefprefixsingle{line }
\setSErefprefixmore{lines }
\setSErefonlypageprefixsingle{p.~}
\setSErefonlypageprefixmore{pp.~}

% Verse
\setstanzaindents{1,1}
\setcounter{stanzaindentsrepetition}{1}
\setlength{\stanzaindentbase}{\parindent}
\AtEveryStanza*{\vspace{0.5\baselineskip}}
\AtEveryStopStanza*{\vspace{0.5\baselineskip}}

% Define languages
\usepackage[
  latin.classic, % Latin without u/v distinction
  italian,
  spanish,
  french,
  ngerman,
  icelandic,
  main=canadian,
]{babel}
\frenchsetup{AutoSpacePunctuation=false}

% Language switching commands
\providecommand{\textlatin}{}
\renewcommand{\textlatin}[2][]{\foreignlanguage{latin}{#2}}
\newenvironment{latin}[2][]{\begin{otherlanguage}{latin}}{\end{otherlanguage}}
\newcommand{\textitalian}[2][]{\foreignlanguage{italian}{#2}}
\newenvironment{italian}[2][]{\begin{otherlanguage}{italian}}{\end{otherlanguage}}
\newcommand{\textfrench}[2][]{\foreignlanguage{french}{#2}}
\newenvironment{french}[2][]{\begin{otherlanguage}{french}}{\end{otherlanguage}}
\newcommand{\textgerman}[2][]{\foreignlanguage{ngerman}{#2}}
\newenvironment{german}[2][]{\begin{otherlanguage}{ngerman}}{\end{otherlanguage}}
\newcommand{\texticelandic}[2][]{\foreignlanguage{icelandic}{#2}}
\newenvironment{icelandic}[2][]{\begin{otherlanguage}{icelandic}}{\end{otherlanguage}}
\let\oritextspanish\textspanish
\AddBabelHook{spanish}{beforeextras}{\renewcommand{\textspanish}{\oritextspanish}}
\AddBabelHook{spanish}{afterextras}{\renewcommand{\textspanish}[2][]{\foreignlanguage{spanish}{##2}}}

% Hyphenation exceptions
\input{ushyphex} % standard English exceptions
\babelhyphenation[canadian]{
  Beck-et
  Gros-se-teste
  manu-script
  me-di-eval
}

\usepackage[prevent-all]{widows-and-orphans}

\recalctypearea % ensure text block matches font metrics

\begin{document}


\frontmatter

% !TEX root = frideswide.tex

\extratitle{
  \begin{centering}
    \vfill
    {\Huge \textsc{Two Priors and a Princess} \par}
    \vfill
  \end{centering}
}

\title{Two Priors and a Princess}
\subtitle{St Frideswide in Twelfth-Century Oxford}
\author{Andrew Dunning\\ \emph{with}\\ Benedicta Ward}
\date{}
\publishers{
  % \includegraphics[height=5em]{images/obp.pdf}
  % \hspace{1em}
  Open Book Publishers

  \textsc{Cambridge}
}

\lowertitleback{{\scriptsize

  \KOMAoptions{parskip=full}

  © 2019 Andrew Dunning

  Translation of Philip of Oxford, \emph{Miracles of St Frideswide} © 2019 Benedicta Ward and Andrew Dunning

  This work is licensed under a Creative Commons Attribution 4.0 International licence (\textsc{CC BY 4.0}). This licence allows you to share, copy, distribute and transmit the text; to adapt the text and to make commercial use of the text providing attribution is made to the authors (but not in any way that suggests that they endorse you or your use of the work). Attribution should include the following information:

    \begin{quote}
    Andrew Dunning with Benedicta Ward, \emph{Two Priors and a Princess: St Frideswide in Twelfth-Century Oxford} (Cambridge: Open Book Publishers, 2019), \url{https://doi.org/10.11647/obp.0183}
    \end{quote}
  
  Further details about \textsc{CC BY} licences are available at \url{https://creativecommons.org/licences/by/4.0/}

  All external links were active at the time of publication unless otherwise stated and have been archived via the Internet Archive Wayback Machine at \url{https://archive.org/}

  Every effort has been made to identify and contact copyright holders and any omission or error will be corrected if notification is made to the publisher.

  \begin{description}
  \tightlist
  \item[ISBN Paperback]
  978-1-78374-785-6
  \item[ISBN Hardback]
  978-1-78374-786-3
  \item[ISBN Digital (PDF)]
  978-1-78374-787-0
  \item[ISBN Digital ebook (epub)]
  978-1-78374-788-7
  \item[ISBN Digital ebook (mobi)]
  978-1-78374-789-4
  \item[ISBN Digital (XML)]
  978-1-78374-790-0
  \item[DOI]
  \url{https://doi.org/10.11647/obp.0183}
  \item[Categories]
  \textsc{BIC: CFP} (Translation and interpretation), \textsc{DSBB} (Literary studies: classical, early and medieval); \textsc{BISAC: LCO017000} (Literary Collections / Medieval), \textsc{LCO009000} (\textsc{LITERARY COLLECTIONS} / European / English, Irish, Scottish, Welsh)
  \end{description}

  All paper used by Open Book Publishers is sourced from \textsc{SFI} (Sustainable Forestry Initiative) accredited mills and the waste is disposed of in an environmentally friendly way.

  \par}
}

\dedication{[dedication]}

\maketitle


\tableofcontents

% !TEX root = frideswide.tex

\chapter{Acknowledgements}



% !TEX root = frideswide.tex

\chapter{Sigla and Abbreviations}

\begin{raggedright}
\setkomafont{descriptionlabel}{}

\begin{description}[leftmargin=!,labelwidth=4em, itemsep=0pt, parsep=0pt]
\item[\emph{B}]
Oxford, Balliol College, \textsc{MS} 228
\item[\emph{C}]
Cambridge, Gonville and Caius College, \textsc{MS} 129/67
\item[\emph{D}]
Oxford, Bodleian Library, \textsc{MS} Digby 177
\item[\emph{G}]
Gotha, Forschungsbibliothek, \textsc{MS} Memb. I 81
\item[\emph{L}]
London, British Library, Lansdowne \textsc{MS} 436
\item[\emph{M}]
Oxford, Bodleian Library, \textsc{MS} Laud misc. 114
\item[\emph{N}]
London, British Library, Cotton \textsc{MS} Nero E. i/2
\item[\emph{P}]
Paris, Bibliothèque nationale de France, \textsc{MS} Latin 5320
\item[\emph{T}]
Cambridge, Trinity College, \textsc{MS} B.14.37
\item[\emph{W}]
Worcester, Cathedral Library, \textsc{MS} Q.86
\end{description}

\begin{description}[leftmargin=!,labelwidth=4em, itemsep=0pt, parsep=0pt]
\item[\emph{add.}]
\textlatin{\emph{addidit}}
\item[\emph{ante corr.}]
\textlatin{\emph{ante correctionem}}
\item[\emph{in marg.}]
\textlatin{\emph{in margine}}
\item[\emph{om.}]
\textlatin{\emph{omisit}}
\item[\emph{sup.~l.}]
\textlatin{\emph{supra lineam}}
\item[Vulg.]
Vulgate
\item[⟨\,\ldots{}\,⟩]
\textlatin{\emph{addendum}}
\end{description}
\end{raggedright}



\mainmatter

% !TEX root = frideswide.tex

\chapter{Introduction}


% !TEX root = frideswide.tex

\part{Cotton-Corpus Legendary: \emph{The Life of St Frideswide the Virgin}}


% !TEX root = frideswide.tex

\part{Robert of Cricklade: \emph{The Life of St Frideswide the Virgin}}


% !TEX root = frideswide.tex

\part{Romsey Legendary: St Frideswide}


% !TEX root = frideswide.tex

\part{Robert of Cricklade: Letter to Benedict of Peterborough}


% !TEX root = frideswide.tex

\part{Philip of Oxford: \emph{The Miracles of St Frideswide}}


\backmatter

% !TEX root = frideswide.tex

\chapter{Bibliography}



\end{document}
